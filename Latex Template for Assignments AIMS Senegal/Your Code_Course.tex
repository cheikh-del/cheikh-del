\documentclass[12pt,a4paper]{article}
\usepackage[latin1]{inputenc}
\usepackage{amsmath}
\usepackage{amsfonts}
\usepackage{amssymb}
\usepackage{graphicx}
\usepackage{multirow,multicol}
\usepackage{pifont,hyperref,lastpage,fancyhdr,movie15,float}

%%%%%%%%%%%%%%%%%%%%%%%%%%%%%%%%%%%%%%%%%%%%%%%%%%%%%%%%%%%%%%%%%%%%%%%%%
%%%%%%%%%%%%%%%%%%%%%%%%%% Footer and Header %%%%%%%%%%%%%%%%%%%%%%%%%%%%
\pagestyle{fancy}                                                     %%%
\fancyhf{} 	                                                      %%%
\lfoot{\tiny\textsf{\includegraphics[width=1.8cm]{AIMSSenegalLogo}    %%%
Po.Box~1418 Mbour-Thies,~phone~{(+221) 33 956 7693},~                 %%%
\url{http://www.aims-senegal.org}}}                                   %%%
\rfoot{\bfseries Page \thepage~of \pageref{LastPage}}                 %%%
\renewcommand{\footrulewidth}{1.pt}\renewcommand{\headrulewidth}{0pt} %%%
%%%%%%%%%%%%%%%%%%%%%%%%%%%%%%%%%%%%%%%%%%%%%%%%%%%%%%%%%%%%%%%%%%%%%%%%%
%%%%%%%%%%%%%%%%%%%%%%%%%%%%%%%%%%%%%%%%%%%%%%%%%%%%%%%%%%%%%%%%%%%%%%%%%

%%%%%%%%%%%%%%%%%%%%%%%%%%%%%%%%%%%%%%%%%%%%%%%%%%%%%%%%%%%%%%%%%%%%%%%%%
%%%%%%% Fill here information about the present assignment %%%%%%%%%%%%%%
\newcommand{\code}{[Your Code]}                                       %%%
\newcommand{\deadline}{[Date, Time]}                                  %%%
\newcommand{\assignment}{assignment [x] ON [Course Name]}			  %%% 
\newcommand{\lecturer}{Lecturer: [Lecturer Name]}                     %%%
%%%%%%%%%%%%%%%%%%%%%%%%%%%%%%%%%%%%%%%%%%%%%%%%%%%%%%%%%%%%%%%%%%%%%%%%%
%%%%%%%%%%%%%%%%%%%%%%%%%%%%%%%%%%%%%%%%%%%%%%%%%%%%%%%%%%%%%%%%%%%%%%%%%

%%%%%%%%%%%%%%%%%%%%%%%%%%%%%%%%%%%%%%%%%%%%%%%%%%%%%%%%%%%%%%%%%%%%%%%%%
%%%%%%%%%%%%%%%%%%%%%%  Title at the first page  %%%%%%%%%%%%%%%%%%%%%%%%
\title{\vspace*{-4cm}\begin{minipage}{\textwidth}                     %%%
\begin{center}                                                        %%%
\begin{tabular}{|c|c|c|}                                              %%%
\hline\multicolumn{3}{|c|}{\bf\scriptsize\MakeUppercase\assignment}\\ %%%
\hline{\small Student's Code}&                                        %%%
\multirow{3}{7cm}{\includegraphics[width=7.5cm,height=2.3cm]{AIMSSenegalLogo}} %%%
& {\small Deadline}\\                                                 %%%
\cline{1-1}\cline{3-3}{\small\bf\code}&&{\small\bf\deadline} \\       %%%
\cline{1-1}\cline{3-3}{\small\today} &&{\small2019-2020}\\            %%%
\hline\multicolumn{3}{|r|}{\scriptsize\lecturer}\\\hline              %%%
\end{tabular}                                                         %%%
\end{center}                                                          %%%
\end{minipage}\hfill\date{}\vspace*{-1cm}}                            %%%
%%%%%%%%%%%%%%%%%%%%%%%%%%%%%%%%%%%%%%%%%%%%%%%%%%%%%%%%%%%%%%%%%%%%%%%%%
%%%%%%%%%%%%%%%%%%%%%%%%%%%%%%%%%%%%%%%%%%%%%%%%%%%%%%%%%%%%%%%%%%%%%%%%%

\newcommand{\K}{\mathbb{K}}
\newcommand{\R}{\mathbb{R}}
\newcommand{\C}{\mathbb{C}}

\newtheorem{theo}{Theorem}
\newtheorem{defi}{Definition}
\newenvironment{proof}[1][Proof.]{\textbf{#1~}}{\ \rule{0.5em}{0.5em}}

\begin{document}
\maketitle\thispagestyle{fancy}

\section{Solution of Exercise 1}
 If we assume that the probability of to choose blue and red are same. Then we have 1/2 for two balls. This is called 'equiprobability'.
\begin{itemize}
	 \item First case, we propose in bucket to have only 3 blues balls in the bucket. The process is, we add 2 balls when the color of ball we choose is blue. And when it's red, we add one red ball. \\




$P_5(3bb) = (\frac{1}{2} \times \frac{3}{4}  \times \frac{2}{5} ) + (\frac{1}{2}  \times \frac{1}{3}  \times \frac{2}{5}) = \frac{2!}{5!} \times(3 + 4)$\\ 

$P_6(3bb) = (\frac{1}{2} \times \frac{1}{3} \times \frac{2}{5} \times \frac{3}{6} ) + (\frac{1}{2}  \times \frac{2}{3} \times \frac{1}{4}  \times \frac{3}{6}) + (\frac{1}{2}  \times \frac{1}{4}  \times \frac{2}{5}  \times \frac{3}{6} = \frac{3!}{6!} \times(3 + 4 + 5) )$ \\

$P_7(3bb) = (\frac{1}{2} \times \frac{2}{3} \times \frac{3}{4} \times \frac{1}{5} \times \frac{4}{7}) + (\frac{1}{2} \times \frac{1}{3} \times \frac{2}{5}  \times \frac{3}{6} \times \frac{4}{7}) + (\frac{1}{2} \times \frac{1}{4} \times \frac{2}{5} \times \frac{3}{6} \times \frac{4}{7})= \frac{4!}{7!} \times(3 + 4 + 5 + 6)$ \\

$P_8(3bb) = (\frac{1}{2} \times \frac{2}{3} \times \frac{3}{4} \times \frac{4}{5} \times \frac{1}{6} \times \frac{5}{8}) + (\frac{1}{2}  \times \frac{2}{3} \times \frac{1}{4}  \times \frac{3}{6} \times \frac{4}{7} \times \frac{5}{8}) + (\frac{1}{2}  \times \frac{1}{4}  \times \frac{2}{5}  \times \frac{3}{6}  \times \frac{4}{7} \times \frac{5}{8}= \frac{5!}{8!}\times(3 + 5 + 7)$ \\

Finally, we obtain $P_9 , P_{10} , P_{11}$ is equal to \\

$P_9(3bb) = \frac{6!}{9!}\times (3 + 5 + 7)$ \\

$P_{10}(3bb) = \frac{7!}{10!}\times (3 + 5 + 7)$ \\

$P_{11}(3bb) = \frac{8!}{11!}\times (3 + 5 + 7)$ \\

Then we can generalise, if n$ \geq8$ the formula is : \\
$P_n(3bb) = \frac{a!}{b!}\times(3 + 5 + 7) = 15\times \frac{a!}{b!}$ , a is the number of red balls in buckets and b is the sum of balls in buckets. \\

The picture below show us the probability tree.


\item In second case, we assume that we have three balls, red, blue and green.
If we choose one color, we have to the one of those two other colors. For example, if I take blue ball, I will add a red or a green ball. \\

$P_4(1red) = (\frac{1}{3} \times \frac{1}{2} \times \frac{1}{4} \times \frac{2}{3}) + (\frac{1}{3}  \times \frac{1}{2} \times \frac{1}{4} \times \frac{1}{3} ) + (\frac{1}{3}  \times \frac{1}{2} \times \frac{1}{4} \times \frac{1}{3}) + (\frac{1}{3}  \times \frac{1}{2} \times \frac{1}{4} \times \frac{2}{3}) = \frac{5!}{8!}\times(3 + 5 + 7)$ \\
\end{itemize}

\begin{theo}
	Nothing is easy in this world!
\end{theo}

\section{Solution of Exercise 2}
















%{\footnotesize
%\begin{thebibliography}{99}
%\bibitem{wik} Wikipedia, \emph{Fundamental theorem of calculus},\\  \href{https://en.wikipedia.org/wiki/Fundamental_theorem_of_calculus}{https://en.wikipedia.org/wiki/Fundamental\_theorem\_of\_calculus}
%\bibitem{fancy}For page style, \href{http://www.emerson.emory.edu/services/latex/latex_129.html}{http://www.emerson.emory.edu/services/latex/latex\_129.html}
%\bibitem{header} Headers and footers, \href{https://www.overleaf.com/learn/latex/Headers_and_footers}{https://www.overleaf.com/learn/latex/Headers\_and\_footers}
%\end{thebibliography}
%}

%%\nocite{*}
%\bibliographystyle{amsalpha}
%%\bibliographystyle{ieeetr-alpha-fr}
%%\bibliographystyle{spr-mp-nameyear-cnd}
%{\footnotesize\bibliography{biblio}}

\end{document}
