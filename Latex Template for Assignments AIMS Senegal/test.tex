\documentclass[12pt,a4paper]{article}
\usepackage[latin1]{inputenc}
\usepackage{amsmath,systeme}
\usepackage{amsfonts}
\usepackage{amssymb}
\usepackage{graphicx}
\usepackage{multirow,multicol}
\usepackage{pifont,hyperref,lastpage,fancyhdr,movie15,float}

%%%%%%%%%%%%%%%%%%%%%%%%%%%%%%%%%%%%%%%%%%%%%%%%%%%%%%%%%%%%%%%%%%%%%%%%%
%%%%%%%%%%%%%%%%%%%%%%%%%% Footer and Header %%%%%%%%%%%%%%%%%%%%%%%%%%%%
\pagestyle{fancy}                                                     %%%
\fancyhf{} 		                                                      %%%
\lfoot{\tiny\textsf{\includegraphics[width=1.8cm]{AIMSSenegalLogo}    %%%
Po.Box~1418 Mbour-Thies,~phone~{(+221) 33 956 7693},~                 %%%
\url{http://www.aims-senegal.org}}}                                   %%%
\rfoot{\bfseries Page \thepage~of \pageref{LastPage}}                 %%%
\renewcommand{\footrulewidth}{1.pt}\renewcommand{\headrulewidth}{0pt} %%%
%%%%%%%%%%%%%%%%%%%%%%%%%%%%%%%%%%%%%%%%%%%%%%%%%%%%%%%%%%%%%%%%%%%%%%%%%
%%%%%%%%%%%%%%%%%%%%%%%%%%%%%%%%%%%%%%%%%%%%%%%%%%%%%%%%%%%%%%%%%%%%%%%%%

%%%%%%%%%%%%%%%%%%%%%%%%%%%%%%%%%%%%%%%%%%%%%%%%%%%%%%%%%%%%%%%%%%%%%%%%%
%%%%%%% Fill here information about the present assignment %%%%%%%%%%%%%%
\newcommand{\code}{[Your Code]}                                       %%%
\newcommand{\deadline}{[Date, Time]}                                  %%%
\newcommand{\assignment}{assignment [x] ON [Course Name]}			  %%% 
\newcommand{\lecturer}{Lecturer: [Lecturer Name]}                     %%%
%%%%%%%%%%%%%%%%%%%%%%%%%%%%%%%%%%%%%%%%%%%%%%%%%%%%%%%%%%%%%%%%%%%%%%%%%
%%%%%%%%%%%%%%%%%%%%%%%%%%%%%%%%%%%%%%%%%%%%%%%%%%%%%%%%%%%%%%%%%%%%%%%%%

%%%%%%%%%%%%%%%%%%%%%%%%%%%%%%%%%%%%%%%%%%%%%%%%%%%%%%%%%%%%%%%%%%%%%%%%%
%%%%%%%%%%%%%%%%%%%%%%  Title at the first page  %%%%%%%%%%%%%%%%%%%%%%%%
\title{\vspace*{-4cm}\begin{minipage}{\textwidth}                     %%%
\begin{center}                                                        %%%
\begin{tabular}{|c|c|c|}                                              %%%
\hline\multicolumn{3}{|c|}{\bf\scriptsize\MakeUppercase\assignment}\\ %%%
\hline{\small Student's Code}&                                        %%%
\multirow{3}{7cm}{\includegraphics[width=7.5cm,height=2.3cm]{AIMSSenegalLogo}} %%%
& {\small Deadline}\\                                                 %%%
\cline{1-1}\cline{3-3}{\small\bf\code}&&{\small\bf\deadline} \\       %%%
\cline{1-1}\cline{3-3}{\small\today} &&{\small2019-2020}\\            %%%
\hline\multicolumn{3}{|r|}{\scriptsize\lecturer}\\\hline              %%%
\end{tabular}                                                         %%%
\end{center}                                                          %%%
\end{minipage}\hfill\date{}\vspace*{-1cm}}                            %%%
%%%%%%%%%%%%%%%%%%%%%%%%%%%%%%%%%%%%%%%%%%%%%%%%%%%%%%%%%%%%%%%%%%%%%%%%%
%%%%%%%%%%%%%%%%%%%%%%%%%%%%%%%%%%%%%%%%%%%%%%%%%%%%%%%%%%%%%%%%%%%%%%%%%

\newcommand{\K}{\mathbb{K}}
\newcommand{\R}{\mathbb{R}}
\newcommand{\C}{\mathbb{C}}

\newtheorem{theo}{Theorem}
\newtheorem{defi}{Definition}
\newenvironment{proof}[1][Proof.]{\textbf{#1~}}{\ \rule{0.5em}{0.5em}}

\begin{document}
\maketitle\thispagestyle{fancy}

\section{Solution of Exercise 1}
Solve the following systems of equations \\
1.
  	
	\begin{gather*}
    \systeme{x + 2y + 3z= 10@\qquad(*), 2x + y - z = 3,-x + 3y + 2z = 5 }
	\end{gather*}
	
	
 If we fixe (1) and try to annul x in (2) and (3). We multiply the equation (1) by -2, and we will have:
	 \begin{gather*}
	 \systeme{-2(x + 2y + 3z= 10)@\qquad(4), 2x + y - z = 3@\qquad(5) }
	 \end{gather*}
	 $\Longrightarrow$\begin{gather*}
	 \systeme{-2x -4y - 6z= -20)@\qquad(4), 2x + y - z = 3@\qquad(5) }
	 \end{gather*}
	 Hence, after sum of equation (4) and (5), the result will be:
	 
	We will have: 
	\[
	\left\{
	\begin{array}{r c l}
	x + 2y + 3z &= 10 \\
	-3y - 7z &= -17\\
	5y + 5z &= 15
	\end{array}
	\right.
	\]
	Secondly, we note the new second equation (4) and the last (5). If we multiply (4) by 5 and (5) by 3, we will have. 
	\[
	\left\{
	\begin{array}{r c l}
	-15y - 35z &= -85\\
	15y + 15z &= 45
	\end{array}
	\right.
	\]
	Then, It will give us z = 2, and if we replace the value of z in (4), hence y = 1. To have the value of x, we can replace in (1). The result is x = 2. The solution is given by:
	S = $\{(2, 1, 2)\}$ \\
2. 
	\[
	\left\{
	\begin{array}{r c l}
	3x - 4y + 5z &= 1 \\
	7x - 2y - 4z  &= 3  \\
	-x - 6y + 14z &= 8 
	\end{array}
	\right.
	\]	

Firstly, we rename the first equation (1), the second (2) and the third (3). If we fixe (1) and try to annule x in (2) and (3). We will have: 

	 \[
	\left\{
	\begin{array}{r c l}
	3x - 4y + 5z &= 1\\
	-22y + 47z &= -2\\
	-22y + 42z &= 25
	\end{array}
	\right.
	\]

   
   
   	\begin{gather*}  	
   	\systeme{3x+y+2z=6@\qquad(*), x+3y+2z=-6, x+y+z=1}    % separation by '\qquad'
   	\end{gather*}
   





\section{Solution of exercice }
Let 
\[
A = 
\begin{bmatrix}
2 & 0 & 0 \\
1 & 3 & -1  \\
2 & 2 & 0
\end{bmatrix}
\]

\begin{enumerate}
	\item[1)] Verify that the matrix $A$ is invertible and compute its inverse $A^{-1}$.
	
	We have to determine the determinant of A 
	\[
	det(A)= \begin{vmatrix}
	2 & 0 & 0 \\
	1 & 3 & -1  \\
	2 & 2 & 0 
	\end{vmatrix} 
	\]
	\qquad 
	\[
	= 2\begin{vmatrix}
	3 & -3  \\
	2 & 0 
	\end{vmatrix} 		
	\] 
	\begin{center}
		$ = 2\times (2) = 4 \neq 0$
	\end{center}
	
	Therefore A is \textbf{inversible}. 
\end{enumerate}	
Now we want to find the inverse of the matrix A.
\\We denote $ A^{c}=(c_{ij})_{1\leq i,j\leq 2} $ the matrix of cofactors of A.
We determine the coefficients of $  A^{c} $:\\
$A^{-1} = \frac{1}{det(A)}\times[cofactorsA]^t$

\[
[cofactors A] =
\begin{bmatrix}
2 & 2 & -4  \\
0 & 0 & 4   \\
0 & -2 & 6
\end{bmatrix}
\]

\[
[cofactors A]^{t} =
\begin{bmatrix}
2 & 0 & 0  \\
2 & 0 & -2   \\
-4 & 4 & 6
\end{bmatrix}
\]
So $A^{-1}$ is equal to:

\[
A^{-1} = \frac{1}{4}\times 
\begin{bmatrix}
2 & 0 & 0  \\
2 & 0 & -2   \\
-4 & 4 & 6
\end{bmatrix}	    
\]

\[
A^{-1} = \begin{bmatrix}
\frac{1}{2} & 0 & 0 \\
\frac{1}{2} & 0 & -2 \\
-1 & 1 & \frac{3}{2} 
\end{bmatrix} 
\]
\begin{enumerate}
	\item[2)]  Compute its characteristic polynomial $P_A(x)$.
\end{enumerate}
We have:
\[
P_{A}(X) = det
\begin{pmatrix}
2-X & 0 & 0\\
1 & 3-X & -1\\
2 & 2 & -X
\end{pmatrix} 
\]
\[=
-X
\begin{bmatrix}
2-X & 0 \\
1 & 3-X 
\end{bmatrix} + 1
\begin{bmatrix}
2-X & 0 \\
2 & 2 
\end{bmatrix} 
\]
\begin{center}
	$= -X(2-X)(3-X) + 2(2-X) $
	$= (2-X)(-X(3-X)+2) \quad$
	$ = (2-X)(-3X + X^2 + 2) \quad$   
	
	
	$ \Rightarrow \quad P_{A}(X) = (2-X)(X-1)(3X+2)$
\end{center}

The eigenvalues of A are $\lambda_{1} = 2, \quad \lambda_{2} = 1 \quad and \quad \lambda_{3} = -\frac{2}{3}$
\begin{enumerate}
	\item[3)] Compute the sets of all eigenvalues and eigenvectors.
	
	The set of eigenvalues is: $Spect_{A} = \{2,1,3\}$. \\
	To find the eigenvectors, we must each eigenvalue and try to find his eigenvectors associated.\\ This is given by: 
	
	If $\lambda_{1} = 2$ $\implies$ $E_{\lambda_{1}} = E_2 = Ker(A-2.I)$ 
	
	Let $x_1, x_2, x_3$ are $\in E_{2}$. Then one has:
	\[
	\left(A-I\right).\begin{pmatrix}
	x_{1}\\
	x_{2}\\
	x_{3}
	\end{pmatrix}
	= \begin{pmatrix}
	0\\
	0\\
	0
	\end{pmatrix}
	\] \quad $ \Rightarrow$\[
	\begin{pmatrix}
	0 & 0 & 0\\
	1 & 1 & -1\\
	2 & 2 & -2
	\end{pmatrix}
	\begin{pmatrix}
	x_{1}\\
	x_{2}\\
	x_{3}
	\end{pmatrix}
	= \begin{pmatrix}
	0\\
	0\\
	0
	\end{pmatrix}
	\]
	\[
	\left\{
	\begin{array}{r c l}
	0 &= 0 \\
	x_1 + x_2 - x_3 &= 0 \\
	2x_1 + 2x_2 - 2x_3 &= 0
	\end{array}
	\right.
	\]
\end{enumerate}















%{\footnotesize
%\begin{thebibliography}{99}
%\bibitem{wik} Wikipedia, \emph{Fundamental theorem of calculus},\\  \href{https://en.wikipedia.org/wiki/Fundamental_theorem_of_calculus}{https://en.wikipedia.org/wiki/Fundamental\_theorem\_of\_calculus}
%\bibitem{fancy}For page style, \href{http://www.emerson.emory.edu/services/latex/latex_129.html}{http://www.emerson.emory.edu/services/latex/latex\_129.html}
%\bibitem{header} Headers and footers, \href{https://www.overleaf.com/learn/latex/Headers_and_footers}{https://www.overleaf.com/learn/latex/Headers\_and\_footers}
%\end{thebibliography}
%}

%%\nocite{*}
%\bibliographystyle{amsalpha}
%%\bibliographystyle{ieeetr-alpha-fr}
%%\bibliographystyle{spr-mp-nameyear-cnd}
%{\footnotesize\bibliography{biblio}}

\end{document}
